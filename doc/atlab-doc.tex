\documentclass[a4paper,11pt]{book}

\usepackage{amsmath}
\usepackage{amssymb}
\usepackage[amssymb]{SIunits}
\usepackage{graphicx}
\usepackage[table]{xcolor}
\usepackage{enumitem}
\usepackage{fancyhdr}
\usepackage[footnotesize,bf]{caption}
\usepackage{sidecap}
% \usepackage{longtable}
% \usepackage{multirow}
% \usepackage{booktabs}
\usepackage[pdfborder={0 0 0}]{hyperref}
% \usepackage{float}

% \usepackage{titlesec}
% \titleformat{\chapter}[hang]{\bf\huge}{\thechapter}{0.5em}{}
% \titlespacing{\chapter}{0pt}{0pt}{50pt}

\usepackage{natbib}
\setlength{\bibsep}{1pt}
\renewcommand{\bibfont}{\footnotesize}

\hoffset -1in
\voffset -1in

\topmargin      10mm
\headheight     10mm
\headsep        10mm
\oddsidemargin  25mm
\textwidth     160mm
\textheight    242mm

%\title{\bf DNS/TMP User Guide}
\title{{\bf AT}\textcolor{black!50}{Lab} Documentation}
\author{Turbulence and Boundary Layers Group $|$ University of Hamburg}

\setcounter{tocdepth}{2}
\setlength{\parindent}{0mm}

\fancyhead[LE,LO]{\footnotesize\leftmark}
\fancyhead[RE,RO]{}

%%%%%%%%%%%%%%%%%%%%%%%%%%%%%%%%%%%%%%%%%%%%%%%%%%%%%%%%%%%%%%%%%%%%%%
\def\pst{\partial_t}
\def\psk{\partial_k}

\def\Re{\mathrm{Re}}
\def\Pr{\mathrm{Pr}}
\def\Sc{\mathrm{Sc}}
\def\Fr{\mathrm{Fr}}
\def\Ro{\mathrm{Ro}}
% \def\Le{\mathrm{Le}}
% \def\Ma{\mathrm{Ma}}
% \def\Da{\mathrm{Da}}

\def\j{\mathbf{j}}
\def\e{\mathbf{e}}

\def\bg{_\mathrm{bg}}
\def\ss{_\mathrm{s}}
\def\sc{_\mathrm{c}}

\newcommand{\pt}[2] {\partial_{#2}{#1}}
\newcommand{\ptp}[2]{\partial_{#2}\left(#1\right)}
% \newcommand{\ptp}[1]{\partial_t\left(#1\right)}
% \newcommand{\pjp}[1]{\partial_j\left(#1\right)}
\newcommand{\dvp}[1]{\nabla\negmedspace\cdot\negthinspace\left(#1\right)}
\newcommand{\dv} [1]{\nabla\negmedspace\cdot\negthinspace#1}
\newcommand{\avr}[1]{\overline{#1}}
\newcommand{\avf}[1]{\widetilde{#1}}


%%%%%%%%%%%%%%%%%%%%%%%%%%%%%%%%%%%%%%%%%%%%%%%%%%%%%%%%%%%%%%%%%%%%%%
\begin{document}

\frontmatter
\pagestyle{empty}
\maketitle
\tableofcontents

\setlength{\parskip}{0.5\baselineskip}
%\include{preface}

\chapter*{Preface}
\addcontentsline{toc}{chapter}{Preface}
\sloppy

ATLab---an acronym for Atmospheric Turbulence Laboratory---is a set of tools whose aim is {\bf to efficiently solve and analyze a particular set of governing equations with a controlled accuracy}. The accuracy can be controlled in different ways: comparing with analytical solutions, including linear stability analysis; grid convergence studies; balance of transport equations, like integral turbulent kinetic energy or local values at specific relevant locations (e.g., at the wall). Resolution can be measured by the ratio between the grid spacing $\Delta x$ and the relevant small scales, like the Kolmogorov scale $\eta$ or the thickness of the diffusion sub-layers next to the wall. For the compact schemes used here, typical values are $\Delta x/\eta\simeq 1-2$; larger values can lead to numerical instability because of the aliasing generated by the non-linear terms. Note that these schemes are non-monotone, but typical out-of-bounds deviations of conserved scalars are below $10^{-6}-10^{-8}$ relative to the mean variations, and this error is therefore negligibly small compared to the typical error associated with the statistical convergence, of the order of $1-5$\%. The statistical convergence can be estimated by varying the sample size of the data set, e.g. varying the domain size along the statistically homogeneous directions. 

The efficiency can be measured in different ways but, ultimate, it should be related with the computational time needed to understand a particular problem with a given accuracy, and so the importance of the controlled accuracy. 

Making the code user-friendly comes after the previous two main priorities: controlled accuracy and efficiency.

Last, the main documentation is the code itself and the examples. This document is only a short introduction to the equations and the tools. The code is continuously under development, so this document is continuously incomplete.

\mainmatter
\pagestyle{fancy}

\part{User Guide}
\chapter{Governing equations}\label{sec:equations}

The code is built to solve the sets of equations described in this section as efficiently as possible. These sets of equations aim to be general enough to cover several problems. As a consequence, one needs to map particular problems to one of these generic cases, and identify the appropriate values of the parameters and defining functions (for instance, $\Re$ or $b^e$). Certain knowledge of the equations is therefore advantageous.

%%%%%%%%%%%%%%%%%%%%%%%%%%%%%%%%%%%%%%%%%%%%%%%%%%%%%%%%%%%%%%%%%%%%%%%%%%%%%%%%
\section{Evolution equations}

\subsection{Boussinesq}

Dynamics and thermodynamics are at most coupled by the buoyancy field, and thermodynamics is not necessary (but can still be used). We consider the momentum equation and the evolution equations for an arbitrary number of scalar fields in the following form:
\begin{subequations}
    \begin{align}
        &\pst v_i = -v_k\psk v_i + \psk\left(\Re^{-1} \nu \psk v_i\right) -\partial_i p'
        +\Fr^{-1} g_i\,b +\Ro^{-1}\,\epsilon_{ijk} f_k\,v_k\;, &&i =1,2,3 \\
        &\pst s_i  = -v_k\psk s_i +\psk \left[(\Re\Sc_i)^{-1} \nu \psk s_i -\psk (\j_i)_k\right] + \omega_i \;, &&i = 1,\ldots\,n
    \end{align}
\end{subequations}
The dynamic variables in these equations are  nondimensionalized by $\rho_0$, $U_0$ and $L_0$. The parameters $\Re$, $\Sc$, $\Fr$ and $\Ro$ need to be provided.
\begin{itemize}
    \item The scalar field $\nu$ represents the kinematic viscosity. In the simplest case, it is equal to 1.
    \item The scalar field $b$ represents the buoyancy, if any. The vector $(g_i)$ gives the direction of the buoyancy force and it is constant.
    \item The vector fields $\j_i$ represent flux terms, if any.
    \item The scalar fields $\omega_i$ represent source terms, if any.
\end{itemize}
These fields are defined in terms of the scalars $s_k$ by functions $\nu^e(s_k)$, $b^e(s_k)$, $\j^e_i(s_k)$ and $\omega^e_i(s_k)$, to be given.
\begin{itemize}
    \item The vector $(f_i)$ gives the direction of the angular velocity of the frame of reference, if any. It is assumed constant.
\end{itemize}
Mass conservation,
\begin{equation}
    \psk v_k=0\;,
\end{equation}
is imposed in terms of the pressure-Poisson equation
\begin{equation}
    \nabla^2 p'=\psk(\ldots)
\end{equation}
with the boundary conditions that result from particularizing the momentum equation at the top and bottom boundaries. Details can be found in \cite{mellado2012factorization}.

\subsubsection{Dimensional Formulation}

A dimensional formulation can be considered by substituting the parameters \texttt{Reynolds}, \texttt{Froude} and \texttt{Rossby} the input file (by default, \texttt{tlab.ini}) with \texttt{Viscosity}, \texttt{Gravity} and \texttt{Coriolis}. The boundary and initial conditions defined in the input file should then be given in dimensional form. 

\subsection{Anelastic}

Dynamics and thermodynamics are coupled by the density. The momentum equation is formulated in terms of the dynamic pressure and the density:
\begin{equation}
    \pst v_i = -v_k\psk v_i + \rho\bg^{-1}\psk\left(\Re^{-1} \mu \psk v_i\right) -\rho\bg^{-1}\partial_i p'
        +\Fr^{-1} g_i(1-\rho/\rho\bg)+\Ro^{-1}\,\epsilon_{ijk} f_k\,v_k\;.
\end{equation}
The evolution equations for the scalars read:
\begin{equation}
    \pst s_i  = -v_k\psk s_i +\rho\bg^{-1}\psk \left[(\Re\Sc_i)^{-1} \mu \psk s_i -\psk (\j_i)_k\right] + \omega_i \;, \qquad i = 1,\ldots\,n
\end{equation}
Symbols are as explained in the Boussinesq case. The scalar field $\mu$ represents the dynamic viscosity. In the simplest case, it is equal to 1.
\par
Mass conservation,
\begin{equation}
    \psk (\rho\bg v_k)=0\;,
\end{equation}
is imposed in terms of the pressure-Poisson equation
\begin{equation}
    \nabla^2 p'=\psk(\rho\bg \ldots)
\end{equation}
with the boundary conditions that result from particularizing the momentum equation at the top and bottom boundaries

\subsection{Compressible}

Dynamics and thermodynamics are fully coupled. The pressure in the momentum equation is the thermodynamic pressure. Mass conservation is expressed in terms of the evolution equation
\begin{equation}
    \pst \rho = -\psk(\rho v_k) 
\end{equation}
instead of a solenoidal constraint.

The variables in these equations are normalized by the reference scales $\mathrm{L}_0$, $\mathrm{U}_0$, $\rho_0$ and $\mathrm{T}_0$, which represent a length, a velocity, a density, and a temperature, respectively. The pressure is normalized by $\rho_0U_0^2$. 

TBD

%%%%%%%%%%%%%%%%%%%%%%%%%%%%%%%%%%%%%%%%%%%%%%%%%%%%%%%%%%%%%%%%%%%%%%%%%%%%%%%%
\section{Thermodynamics}

We consider $n\sc$ components (or species) with mass fractions $\zeta_i$.

\subsection{Compressible}
The thermal equation of state is implemented as
\begin{equation}
    p = \rho R T \;.
\end{equation}
The scalar field 
\begin{equation}
    R =\sum^{n\sc}_1 R_i\zeta_i
\end{equation}
is the specific gas constant of the mixture.

The caloric equation of state is implemented as
\begin{equation}
    h = \sum^{n\sc}_1 h_{i} \zeta_i \;,\qquad h_{i} = \Delta h^0_i + \int^{T}_{T_0}
    c_{pi}(T) dT \;.
\end{equation}

Each species has a specific heat capacity $c_{pi}$ and a specific gas constant $R_i=\mathcal{R}/W_i$,  where  $\mathcal{R}$ is the universal gas constant and $W_i$ the molar mass of the species. They are nondimensionalized by the reference values $c_{p0}$ and $R_0=\mathcal{R}/W_0$, respectively, which are the values of one of the species. %They are defined in the procedure \texttt{Thermodynamics\_Initialize}. 

The thermodynamic variables are nondimensionalized as in evolution equations, i.e., the reference scales $\mathrm{L}_0$, $\mathrm{U}_0$, $\rho_0$ and $\mathrm{T}_0$, which represent a length, a velocity, a density, and a temperature, respectively. The pressure is normalized by $\rho_0U_0^2$. Thermal energy variables are normalized with $c_{p0}T_0$, where $c_{p0}$ is a reference specific heat capacity at constant pressure.

TBD

\subsubsection{Dimensional Formulation}

In this case of a multi-species, a dimensional formulation can be considered by setting the input parameter \texttt{nondimensional} equal to \texttt{.false.} in the block \texttt{[Thermodynamics]} of the input file (by default, \texttt{tlab.ini}). 

\subsection{Anelastic}
The thermal equation of state is implemented as
\begin{equation}
    p\bg = \rho R T \;.
\end{equation}
The thermodynamic variables are nondimensionalized by $\rho_0$, $T_0$ and $R_0$, such that $p_0=\rho_0R_0T_0$. The default reference values are $p_0=10^5$~Pa and $T_0=298$~K. Thermal energy variables are normalized with $c_{p0}T_0$, where $c_{p0}$ is a reference specific heat capacity at constant pressure.

The caloric equation of state is formulated in terms of the static energy
\begin{equation}
    h =\sum^{n\sc}_1 h_{i} \zeta_i + \frac{\gamma_0-1}{\gamma_0}H^{-1}(x_3-x_{3,0}) \;.
\end{equation}
The scalar field $c$ is the specific heat capacity, a function of the scalars $s_i$. The parameter
\begin{equation}
    H = \frac{R_0T_0}{gL_0}
\end{equation}
is a nondimensional scale height, or the inverse of a nondimensional gravity, to be provided. The parameter
\begin{equation}
    R_0/c_{p0} =(\gamma_0-1)/\gamma_0 \;,
\end{equation}
a conversion factor between gas constants and heat capacities (thermal equation of state and caloric equation of state). We refer to it in the code as \texttt{GRATIO}. 

The background profiles $\{\rho\bg,\, p\bg,\, T\bg,\, \zeta_{i,\mathrm{ref}}\}$ correspond to a state of thermodynamic and hydrostatic equilibrium. The code solves the system of equations
\begin{subequations}
    \begin{align}
        &\partial_3\,p\bg=-H^{-1}\, g_3\,\rho\bg\;,\qquad p\bg|_{x_3=x_{3,0}}=p_{\mathrm{bg},0}\;,\\
        &p\bg  = \rho\bg R\bg T\bg \;.
    \end{align}
\end{subequations}
$\mathbf{g}$ is defined opposite to the gravitational acceleration (the problem is formulated in terms of the buoyancy). The two equations above relate 4 thermodynamic variables and we need two additional constraints. Typically, we impose the background profile of static energy (enthalpy plus potential energy) and the composition. If there is only one species, then $R\bg=1$ and we only need one additional constraint. 

Currently implemented only for air-water mixtures. In this case, the first scalar is the energy variable and the remaining scalars are the composition (e.g., total water specific humidity and liquid water specific humidity).

\subsubsection{Dimensional Formulation}

A dimensional formulation can be considered by setting the parameter \texttt{nondimensional} equal to \texttt{.false.}, which sets \texttt{GRATIO} equal to 1

\subsection{Mixtures}

We consider $n\sc$ components (or species) with mass fractions $\zeta_i$. These need not be equal to the prognostic scalar variables $s_k$, and the relationship between them $\zeta_i=\zeta_i^e(s_k)$ needs to be given.

In the generic case, we choose
\begin{equation}
    \zeta_i=s_i\;,\quad i=1,\ldots,n\sc-1\;,
\end{equation}
and the last component has a mass fraction
\begin{equation}
    \zeta_{n\sc} = 1-\sum_1^{n\sc-1}\zeta_i \;.
\end{equation} 
The gas constant can then be written as 
\begin{equation}
    R=\sum_1^{n\sc}\zeta_iR_i=\sum_1^{n\sc-1}(R_i-R_N) \;,
\end{equation}
and similarly for other thermodynamic variables.

Particular cases different from this one occur for instance when we consider chemical equilibrium or phase equilibrium, or when different combinations of mass fractions might be preferable to better represent conservation properties.

%For instance, in the case of moist thermodynamics with phase equilibrium (saturation adjustment), only the total water content is a prognostic variables and the partition into liquid and vapor is done assuming phase equilibrium.

\subsubsection{Air-water mixtures in anelastic formulations}

We consider the total water specific humidity as prognostic variable. If necessary, we also consider the liquid water specific humidity, which can be diagnostic in the case of phase equilibrium, or prognostic otherwise.


% %%%%%%%%%%%%%%%%%%%%%%%%%%%%%%%%%%%%%%%%%%%%%%%%%%%%%%%%%%%%%%%%%%%%%%%%%%%%%%%%
% \section{Dimensional formulations}


% \include{bcs}
% \include{code}
% \include{grid}
\include{postprocessing}

\part{Technical Guide}
% \chapter{Numerical Algorithms}\label{sec:numerics}

The system of equations is written as
\begin{subequations}
    \begin{align}
        &\pst \q = \mathbf{F}_q(\q,\,\s,\,t) \;,\\
        &\pst \s = \mathbf{F}_s(\q,\,\s,\,t) \;.
    \end{align}
\end{subequations}
where $\q$ and $\s$ are the vector vectors of flow and scalar prognostic variables. For the incompressible formulations, we have
\begin{subequations}
    \begin{align}
        &\q = (u_1,\,u_2,\,u_3)^T \;,\\
        &\s = (s_1,\,s_2,\,\ldots)^T\;.
    \end{align}
\end{subequations}
The code uses the method of lines, so that the algorithm is a combination of different spatial operators that calculate the right-hand side of the equations and a time marching scheme. 

%%%%%%%%%%%%%%%%%%%%%%%%%%%%%%%%%%%%%%%%%%%%%%%%%%%%%%%%%%%%%%%%%%%%%%%%%%%%%%%%
%%%%%%%%%%%%%%%%%%%%%%%%%%%%%%%%%%%%%%%%%%%%%%%%%%%%%%%%%%%%%%%%%%%%%%%%%%%%%%%%
\section{Spatial operators}

Spatial operators are based on finite difference methods (FDM). There are two levels of routines. The low-level libraries contain the basic algorithms and are explained in this section. It consists of the FDM kernel library {\tt finitedifferences} and three-dimensional operators library {\tt operators}. The high-level library {\tt mappings} is composed of routines that are just a combination of the low-level routines.

\subsection{Derivatives}\label{sec:fdm}

See file {\tt operators/opr\_partial}. 

Spatial derivatives are calculated using fourth- or sixth-order compact Pad\'{e} schemes as described by \cite{lele1992compact} and \cite{lamballais2011straightforward} for uniform grids and extended by \cite{shukla2005derivation} for non-uniform grids. The kernels of the specific algorithms are in the library {\tt finitedifferences}.


\subsection{Fourier transform}

See file {\tt operators/opr\_fourier}. 

It is based on the FFTW library \citep{frigo2005design}.

% The sequence of transformations is $Ox\rightarrow Oy\rightarrow Oz$. The transformed field contains the Nyquist frequency, so it needs an array {\tt(imax\_total/2+1)}$\times${\tt jmax\_total}$\times${\tt kmax\_total} of complex numbers.

% Given the scalar field $s$, the power spectral density $\{E_0,\,E_1,\,\ldots,\,E_{N/2}\}$ is normalized such that
% \begin{equation}
% \langle s^2\rangle = E_0+2\sum_0^{N/2-1}E_n+E_{N/2} \;.
% \end{equation}
% The mean value is typically removed, such that the left-hand side is $s^2_\text{rms}$. The Nyquist frequency energy content $E_{N/2}$ is not written to disk, only the $N/2$ values $\{E_0,\,E_1,\,\ldots,\,E_{N/2-1}\}$.

\subsection{Poisson equation}

See file {\tt operators/opr\_elliptic}. 

Given the scalar field $s$, obtain the scalar field $f$ such that
\begin{equation}
  \nabla^2 f= s \;,
\end{equation}
complemented with appropriate boundary conditions.  The current version only handles cases with periodic boundary conditions along $Ox$ and $Oy$. It performs a Fourier decomposition along these two directions, to obtain the a set of finite difference equations along $Oz$ of the form
\begin{equation}
  \delta_x \delta_x \mathbf{f}|_j - (\lambda_1/h)^2\mathbf{f}|_j=\mathbf{s}|_j
  \;,\qquad j=2,\ldots,n-1 \;,
\end{equation}
$\lambda_1\in\mathbb{R}$, where boundary conditions need to be provided at $j=1$ and $j=n$.  The algorithm is described in \cite{mellado2012factorization}. We can also consider the case in which the second-order derivative is implemented in terms of the $\delta_{xx}$ FDM operator, not only the $\delta_x\delta_x$ FDM operator. These routines are in the source file {\tt operators/opr\_odes}.

\subsection{Helmholtz equation}
\label{sec:helmholtz}

See file {\tt operators/opr\_elliptic}. 

Given the scalar field $s$, obtain the scalar field $f$ such that
\begin{equation}
\nabla^2 f + \alpha f= s \;,
\end{equation}
complemented with appropriate boundary conditions. The current version only handles cases with periodic boundary conditions along $Ox$ and $Oy$. The algorithm is similar to that used for the Poisson equation. It performs a Fourier decomposition along these two directions, to obtain the a set of finite difference equations along $Oz$ of the form
\begin{equation}
  \delta_x \delta_x \mathbf{f}|_j - (\lambda_2/h^2-\alpha)\mathbf{f}|_j=\mathbf{s}|_j
  \;,\qquad j=2,\ldots,n-1 \;,
\end{equation}
$\lambda_2\in\mathbb{R}$, where boundary conditions need to be provided at $j=1$ and $j=n$.

%%%%%%%%%%%%%%%%%%%%%%%%%%%%%%%%%%%%%%%%%%%%%%%%%%%%%%%%%%%%%%%%%%%%%%%%%%%%%%%%
%%%%%%%%%%%%%%%%%%%%%%%%%%%%%%%%%%%%%%%%%%%%%%%%%%%%%%%%%%%%%%%%%%%%%%%%%%%%%%%%
\section{Time marching schemes}

See file {\tt tools/simulation/timemarching}. 

The time advancement is based on Runge-Kutta methods (RKM).

\subsection{Explicit schemes}

We can use three- or five-stages, low-storage RKM that gives third- or fourth-order accurate temporal integration, respectively \citep{williamson1980low,carpenter1994fourth}. The stability properties for the biased finite difference schemes are considered in \cite{carpenter1993stability}. The incompressible formulation follows \cite{williamson1980low}.

\subsection{Implicit schemes}

An implicit treatment of the diffusive terms in the incompressible case follows \cite{spalart1991spectral}. TBD.

% The dissipation and dispersion error maps corresponding to the third-order implicit Runge-Kutta scheme are shown in figure~\ref{fig:rkm3implicit}. The algorithm is unconditionally stable but we need to control accuracy of the diffusion operator for which it is used. The reference value $\textrm{CFL}_d=1.7$ as it gets most of the eigenvalues within the 1\%-error region.


\chapter{Memory management}

See file \texttt{base/tlab\_memory}.

Main arrays are allocated during initialization. Intermediate variables are to be stored in \texttt{tmp} and not in scratch arrays. Scratch arrays can always be used in low level procedures (e.g., \texttt{finitedifferences}), and it is better not to use them in high level procedures (e.g., \texttt{operators}).

\begin{table}[!h]
    \footnotesize
    \renewcommand{\arraystretch}{1.2}
    \centering
    \rowcolors{1}{white}{gray!25}
    \begin{tabular}{lll}
        \hline
        array & size & content \\
        \hline
        \texttt{x}      & number of points in $Ox$  & $x$-coordinate information        \\
        \texttt{y}      & number of points in $Oy$  & $y$-coordinate information        \\
        \texttt{z}      & number of points in $Oz$  & $z$-coordinate information        \\
        \texttt{g}      & number of points in each direction $\times$ number of arrays for numerics           &  finite differences information        \\
        \texttt{q}      & number of points $\times$ number of flow fields                           & flow variables        \\
        \texttt{s}      & number of points $\times$ number of scalar fields                         & scalar variables      \\
        \texttt{txc}    & extended number of points $\times$ number of temporary fields             & temporary variables   \\
        \texttt{wrk3d}  & extended number of points                                                 & scratch               \\
        \texttt{wrk2d}  & maximum number of points in 2D planes $\times$ number of scratch planes   & scratch               \\
        \texttt{wrk1d}  & maximum number of points in 1D lines $\times$ number of scratch lines     & scratch               \\
        \hline
    \end{tabular}
    \caption{Main arrays and their sizes in terms of the number of points. The first 4 are derived types.}
\end{table}

Pointers are also defined during initialization to access these memory spaces with arrays of different shape and even different type, more specifically, complex type needed in Fourier decomposition. See corresponding procedures in \texttt{base/tlab\_memory}.
% \include{decomposition}
% \include{scaling}
\chapter{Profiling}\label{sec:profiling}

The diagrams shown below have been constructed with \texttt{gprof2dot.py}\footnote{\url{https://github.com/jrfonseca/gprof2dot}}.

\begin{figure}[!h]
  \centering
  \includegraphics[clip,width=\textwidth]{fig-profiling08.pdf}
  \caption{Profiling diagram of \texttt{examples/Case08} running 10 iterations in serial mode. Profiling data obtained from \texttt{gfortran -pg} and processed with \texttt{gprof}, running the command \texttt{gprof path/to/your/executable | gprof2dot --color-nodes-by-selftime | dot -Tpdf -o output.pdf}.}
\end{figure}

\newpage

\begin{figure}[!h]
  \centering
  \includegraphics[clip,width=0.9\textheight,angle=90]{fig-profiling65.pdf}
  \caption{Profiling diagram of \texttt{examples/Case65} running 10 iterations in serial mode. Profiling data obtained from \texttt{gfortran -pg} and processed with \texttt{gprof}, running the command \texttt{gprof path/to/your/executable | gprof2dot --color-nodes-by-selftime | dot -Tpdf -o output.pdf}.}
\end{figure}

% \caption{Profiling diagram of modified \texttt{examples/Case44} (768 cube) running 25 iterations in parallel mode with 48 tasks (1 node on juwels).}


\backmatter
\bibliographystyle{plainnat}
\bibliography{atlab-doc.bib}

\end{document}
